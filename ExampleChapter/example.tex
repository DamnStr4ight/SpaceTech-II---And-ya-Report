\documentclass[a4paper]{report}
\usepackage[utf8]{inputenc}
\usepackage{amsmath}
\usepackage{standalone}
\usepackage{graphicx}
\usepackage{hyperref}
\usepackage{caption}
\usepackage[colorinlistoftodos]{todonotes} %\todo, \missingfigure
\usepackage[norsk]{babel}
\usepackage{units}
\usepackage[amssymb]{SIunits}
\usepackage{placeins} %To use \FloatBarrier
\newcommand{\tab}[1]{\hspace{.2\textwidth}\rlap{#1}}

%-----------------------------------------------------------------
%Når man skirver her, kommer det ikke i teksten

\begin{document}
    \listoftodos

    \chapter{Eksempelkapittel}
    
    \begin{figure}[h!]
        \centering
        \includegraphics[width=1\textwidth]{figures/eksempel.jpg}
        \caption{
            Dette er teksten som kommer under bildet. Her kan jeg også sitere bibliografien\cite{databladtemp}.
        } 
        \label{fig:eks-stempel}
    \end{figure}
    
    Juster størrelsen på bildet ved å endre \textit{width}.
    I teksten kan det være at jeg vil referere til en figur. Det gjøres slik ved å bruke figurens ``label-nøkkel'' slik: Figur \ref{fig:eks-stempel}. Her viste jeg også hvordan anførselstegn \textbf{MÅ} brukes. Noen spesialtegn har andre betydninger i latex, slik som \&. Latex bruker dette tegnet til å dele opp tekst i kolonner (eksempelvis vist med tabell \ref{table:gps-points}). For å vise de faktiske tegnene settes en backslash foran.
    
    Generell regel i starten:
    \begin{itemize}
        \item{Punkt én: Finn ut hva du vil lage, feks en punktliste.}
        \item{Punkt to: Søk på google etter eksempler.}
        \item{Punkt tre: Tilpass eksempelet ditt bruk.}
    \end{itemize}
    
    
    
    
    
    
    
    
    
    
    
    Samme hvor mange linjeskift jeg har i koden blir det tolket som et nytt avsnitt. Man kan også tvinge frem et linjeskift med dobbel backslash.\\
    Da vil teksten begynne på neste linje, uten å bli indentert. Jeg kan også fortsette på neste side, slik. \newpage
    
    Det finnes noen lure hurtigtaster. Disse finner du her: \url{https://no.sharelatex.com/learn/Kb/Hotkeys}
    
    Et annet lurt verktøy er \textit{todo's}. Da kan vi lage stilige påminnelser, som vi senere kan se i en liste. I tillegg kan vi skjule alle todo-meldinger. \todo{Husk: Todo-meldinger kan slås av og på ved å skrive \textit{disable} i klammeparantesene der todonotes importeres} Mer todo-triksing finnes her \url{http://tug.ctan.org/macros/latex/contrib/todonotes/todonotes.pdf}
    
    \missingfigure{Dette er en påminnelse om at vi må huske å sette inn en figur her.}
    Her skal jeg sitere en artikkel som ligger i General/references.\cite{lenz1990review} Hver referanse har en nøkkel som vi velger selv. \todo[inline]{Merk: hvis kun ett kapittel kompileres ser det ikke bibliografien, og referansene viser [?]. Kun når main.tex kompileres vises referanser skikkelig. Sitering og referering gjøres med nøkler, i dette tilfellet, slik som lenz1990review. Slik oppdateres nummerering av kilder automatisk.}
    Hvis vi vil lage forskjellige farger på gjøremålene, gjøres det slik.\todo[color=green!40]{Denne er grønn med intensitet 40}
    


    Noen ganger vil vi vise to bilder ved siden av hverandre:
    \begin{figure}[h!]
        \begin{minipage}{0.5\textwidth}
            \centering
            \includegraphics[width=0.9\textwidth]{figures/specimen1.jpg}
            \caption{Menneske 1} 
            \label{fig:johan}
        \end{minipage}
        \begin{minipage}{0.5\textwidth}
            \centering
            \includegraphics[width=0.8\textwidth]{figures/specimen2.jpg}
            \caption{Menneske 2} 
            \label{fig:helene}
        \end{minipage}
    \end{figure}
    
    Andre ganger vil vi lage en tabell. Her benyttes \&-tegnet til å definere kolonner. Indenteringen som er gjort i koden har ingen innvirkning på tabellen, men gjør koden mer oversiktlig. Bokstavene l, c og r øverst bestemmer om teksten i kolonnen skal justeres mot venstre, høyre eller sentreres.
    
    \begin{center}
        \captionof{table}{GPS-punkter}
        \begin{tabular}{ | l | c | c | c | r | }
            \hline
            Point id &       X &       Y & Weight & 		 N\\
            \hline
            \hline
            HAVSTEIN &       0 &       0 &		1 &		39.632\\
            HMV1     &  794.17 &  1015.5 &		1 &		39.572\\
            HMV1EX   &     798 &  1003.6 &		1 &		39.565\\
            MOHOLT   &  296.66 &  2688.4 &		1 &		 39.52\\
            ST46     &  2831.2 &  2797.7 &		4 &		39.434\\
            STUBBAN  & -1155.5 &  1793.5 &		1 &		39.606\\
            \hline
        \end{tabular}
        \label{table:gps-points}
    \end{center}

    Latex gjør det også veldig enkelt å skrive komplekse matematiske formler. For enkle formler bruker vi \$ på begge sider. $E = MC^{2}$
    Mer avanserte formler med referering lages som i formel \ref{eq:ncr}.
    
    \begin{equation}
        \label{eq:ncr}
        \left(\!
        \begin{array}{c}
          n \\
          r
        \end{array}
        \!\right) = \frac{n!}{r!(n-r)!}
    \end{equation}
    
    \bibliographystyle{ieeetr}
    \bibliography{General/references}

\end{document}
